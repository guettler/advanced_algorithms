\documentclass[11pt, notitlepage]{scrartcl}
\usepackage{graphicx}    % needed for including graphics e.g. EPS, PS
\usepackage{amsmath}
\usepackage[T1]{fontenc}
%\usepackage[utf8]{inputenc}
\usepackage[latin1]{inputenc}
\usepackage[german]{babel}
\usepackage{subfigure}
\usepackage{multirow}
\usepackage{multicol}
\usepackage{booktabs} % for midrule
\usepackage{colortbl}
\usepackage{color}
\usepackage{hyperref}
\definecolor{Gray}{gray}{0.6}

\usepackage{tabularx}          
\newcolumntype{C}[1]{>{\centering\arraybackslash}p{#1}} 
\newcolumntype{R}[1]{>{\raggedleft\arraybackslash}p{#1}} 

\addtokomafont{captionlabel}{\bfseries} %references bf

\topmargin -1.5cm        % read Lamport p.163
\oddsidemargin -0.04cm   % read Lamport p.163
\evensidemargin -0.04cm  % same as oddsidemargin but for left-hand pages
\textwidth 16.59cm
\textheight 24cm 
\parskip 7.2pt           % sets spacing between paragraphs
\parindent 0pt     % sets leading space for paragraphs

\usepackage{fancyhdr}
\pagestyle{fancy}
\fancyhf{}



%\renewcommand{\headrulewidth}{0.5pt}
%\fancyfoot[C]{\thepage}
%\renewcommand{\footrulewidth}{0.5pt}
% folder for images
%\graphicspath{{./img/}}
\begin{document}         


\title{Advanced Algorithms for Bioinformatics} 
\subtitle{Exercise 2: Read mapping with QUASAR}
%\author{ }
%\author{}
\author{Group 5: N. G"uttler, K. Liebers, F. Mattes} % lexicogrphic sorted
\maketitle

%%%%%%%%%%%%%%%%%%%%%%%%%%%%%%%%%%%%%%%%%%%%%%%%%%%%%%%%%%%%%%%%%%%%%%%%%%%%%
%%%%%%%%%%%%%%%%%%%%%%%%%%%%%%%%%%%%%%%%%%%%%%%%%%%%%%%%%%%%%%%%%%%%%%%%%%%%%
%%%%%%%%%%%%%%%%%%%%%%%%%%%%%%%%%%%%%%%%%%%%%%%%%%%%%%%%%%%%%%%%%%%%%%%%%%%%%
\section{Implementation}

% #######################################################################################
\section{Results/Observations}
\subsection{Comparison with the runtime from Exercise 1}
In order to show the impact of filtering, the program 'exercise2.cpp' was executed with the same parameters as 'exercise1' ($k=0$, Ukkonen trick = on) and the block length was set to be equal to the read length. After a few tries with different values for the q-gram length, $q=8$ was chosen since it seemed to be suitable for all data sets.

The following table shows the new  ($3^{rd}$ row) and old runtimes.


\begin{center}
\begin{tabular}{c|c|c||c|c|c}
\toprule
\multirow{2}*{Reads file's name} &  \multicolumn{3}{c|}{Running time [sec]}& \multirow{2}*{Nr. of occurrences} & Nr. of verifications\\
\cline{2-4}
&exercise1.cpp& exercise2.cpp & Razers&&(in exercise2.cpp)\\
\hline
{50\_100}&13.60&\textbf{1.49} &11.79 &31&31\\

\hline
{50\_1k}& 136.47& \textbf{2.47}&12.42&289&289\\
 \hline
100\_100& 13.79&\textbf{1.54}&11.95&16&16\\

\hline
100\_1k&136.68 &\textbf{2.72}&13.23&189&190\\
 \hline
400\_100& 15.66&\textbf{1.61}&12.24 &11&13\\
\hline
400\_1k& 137.23&\textbf{3.63}&16.19 &54&72\\
\bottomrule
\end{tabular}
\end{center}

Program 'exercise2' was executed on the same linux machine \textit{andorra}\footnote{andorra.imp.fu-berlin.de} as 'exercise1' so that the remarkably reduction of the runtime is based only on the filtering process. The number of the semi-global aligner calls, i.e., the number of verifications was in the most cases equal to the number of occurrences. 

Setting $q=9$ the number of verification for the last data set is reduced from 72 to 55, reducing also the runtime to $3.47s$. For the other cases, where the number of verifications is already equal to the number of occurrences, an increase of the value of $q$ do not improve the runtime but slows down the procedure, since the number of permutations of a string of the length $q$ increase exponentially. E.g. with $q=9$ the program's runtime on the first data set amounts to $2.70s$.



\subsection{Larger datasets}
Test different values of $k$, $q$ and $b$.

%%%%%%%%%%%%%%%%%%%%%%%%%%%%%%%%%%%%%%%%%%%%%%%%%%%%%%%%%%%%%%%%%%%%%%%%%%%%%
%%%%%%%%%%%%%%%%%%%%%%%%%%%%%%%%%%%%%%%%%%%%%%%%%%%%%%%%%%%%%%%%%%%%%%%%%%%%%
%%%%%%%%%%%%%%%%%%%%%%%%%%%%%%%%%%%%%%%%%%%%%%%%%%%%%%%%%%%%%%%%%%%%%%%%%%%%%
\end{document}

%% 400_ik 0 2 400 -> 5692.45s
%random10M_reads100_100k.fasta 0 12 100 -> 23,09s 17481 b=1000 10,31	k=5 27,98
%random10M_reads100_100k.fasta 0 10 100 -> 30,49s 17481				11.39		29,30
%random10M_reads100_100k.fasta 0 8 100 -> 124,00s 17481				38.08	55,69
%
%random10_reads100_k  0 10 100 ->30.33s







% windows 12 Q
%400_1k 0 8 400 -> 21,47