\documentclass[11pt, notitlepage]{scrartcl}
\usepackage{graphicx}    % needed for including graphics e.g. EPS, PS
\usepackage{amsmath}
\usepackage[T1]{fontenc}
%\usepackage[utf8]{inputenc}
\usepackage[latin1]{inputenc}
\usepackage[german]{babel}
\usepackage{subfigure}
\usepackage{multirow} 
\usepackage{multicol}
\usepackage{booktabs} % for midrule
\usepackage{colortbl}
\usepackage{color}
\usepackage{hyperref}
\definecolor{Gray}{gray}{0.6}

\usepackage{tabularx}          
\newcolumntype{C}[1]{>{\centering\arraybackslash}p{#1}} 
\newcolumntype{R}[1]{>{\raggedleft\arraybackslash}p{#1}} 

\addtokomafont{captionlabel}{\bfseries} %references bf

\topmargin -1.5cm        % read Lamport p.163
\oddsidemargin -0.04cm   % read Lamport p.163
\evensidemargin -0.04cm  % same as oddsidemargin but for left-hand pages
\textwidth 16.59cm
\textheight 24cm 
\parskip 7.2pt           % sets spacing between paragraphs
\parindent 0pt     % sets leading space for paragraphs

%\usepackage{fancyhdr}
%\pagestyle{fancy}
%\fancyhf{}



%\renewcommand{\headrulewidth}{0.5pt}
%\fancyfoot[C]{\thepage}
%\renewcommand{\footrulewidth}{0.5pt}
% folder for images
%\graphicspath{{./img/}}
\begin{document}         


\title{Advanced Algorithms for Bioinformatics} 
\subtitle{Exercise 3: Compression with BWT}
%\author{ }
%\author{}
\author{Group 5: N. G"uttler, K. Liebers, F. Mattes} % lexicogrphic sorted
\maketitle

%%%%%%%%%%%%%%%%%%%%%%%%%%%%%%%%%%%%%%%%%%%%%%%%%%%%%%%%%%%%%%%%%%%%%%%%%%%%%
%%%%%%%%%%%%%%%%%%%%%%%%%%%%%%%%%%%%%%%%%%%%%%%%%%%%%%%%%%%%%%%%%%%%%%%%%%%%%
%%%%%%%%%%%%%%%%%%%%%%%%%%%%%%%%%%%%%%%%%%%%%%%%%%%%%%%%%%%%%%%%%%%%%%%%%%%%%
\section{Implementation}
%\textbf{q-gram index}\\


Tasks: (nur als Anhaltspunkte, d.h. am Ende weg damit!)

Mode c
\begin{itemize}
	\item   reads a single sequence from a fasta file
	   \item calculates the BWT of that sequence
    \item implements move-to-front encoding and Huffman coding to compress the BWT
    writes the Huffman code into an outfile 
	
\end{itemize}

Mode x
\begin{itemize}
	\item     reads a file containing the BWT compression of some sequence
	\item writes the uncompressed sequence into a fasta outfile. 
\end{itemize}

\textbf{Short introduction (TBD by?)}\\
\subsection{Mode c}




  
\textbf{BWT and move-to-front}\\
With the library \textit{seqan} the suffix array of the given sequence was obtain. For this the function \textit{createSuffixArray() }with option \textit{Skew7} was used. The BWT derivation from the suffix array as well as the move-to-front algorithm were implemented according to the lecture's script.

\emph{Remark:} on linux we could not import \textit{seqan} properly(\#include <seqan/index.h>), without using some IDE. Therefore warning messages or even problems by compiling are unfortunately possible.


\textbf{Huffman code}\\
By scanning the BWT the used characters (i.e. the alphabet) and their frequencies were determined. On the basis of these data the Huffman code of the R array was calculated.\\
The conceptual Huffman tree were build using a defined data structure '\textit{node}' and the STL container list, as priority queue.

With the information that for $N$ leafs, a binary tree has $2N-1$ nodes, the Huffman tree can be stored in an array of '\textit{nodes}'. The first $N$ entries contains the character carrying leafs; the remaining entries were filled during the Huffman algorithm. The last entry of the array correspond to the tree's root.

\textit{Struct} \textit{node} contains all information needed to establish the relationship between the nodes of the array. E.g. frequency, pointer to the children (left, right), parent and own id w.r.t. the index of the node's array and so on. Once the algorithm is done (i.e. only one node remains in the queue) all entries of the array has been processed. Then, the sequence of bit codes of each character can be obtained inverting the path computed from the leafs to the root. (For more details see code).

The container list was selected, because it allows to sort the elements according to some argument, in this case the according to the frequency value saved in each node. Accessing to the front of the list, the nodes with the 2 smallest frequencies could be obtain easily. \\
After the insertion of the new node, the queue is sorted again.

% #######################################################################################
\subsection{Mode x}
\subsubsection{Huffman Code -> R Decoding}
short description 2-3 sentences TBD by Kurt, may be unnecessary.
\subsubsection{L-to-F-Mapping aka BWT to Original Text Decoding}
\subsubsection{Move-to-Front Decoding}

% #######################################################################################
\section{Results/Observations}
\subsection*{Comparison of the compression rate}
The goal of this programming assignment was to create a program able to transform a given text - e.g. a genome - into a
datastructure with which search operations are still possible. The first step in this transformation is to create a
suffix array and then create a BWT representation form it. After that, the task was to also compress this representation
and store it to the hard-drive to save disk space. Compression was conducted in two steps:
\begin{enumerate}
    \item Using move-to-front encoding to create a sequence of integers instead of chars to save space.
    \item Using Huffman encoding to optimize bit usage for the integer sequence from step 1.
\end{enumerate}
Encoding and compressed results are stored to the hard disk.\\\\
On the other hand, the invert route was constructed, too. The program in mode `x` is able to read in the binary
compressed file with the data. Using the data all steps (compression and encoding) are reverted and the result is
checked for test purposes. \\\\
In the following we present an overview of the compression of our program and wide-spread compression tools. This way
effectiveness is checked and compared. \\\\
\begin{center}
\begin{tabular}{c|c|c|}
\toprule
\multirow{2}*{Method's name} &  \multirow{2}*{Original file size} & \multirow{2}*{Compressed file size} \\
\hline
{Assignment program} & 9.7 MB & 0.0 MB\\
\hline
{WinZIP} & 9.7 MB & 0.0 MB \\
\hline
{bzip2} & 9.7 MB & 2.8 MB \\
\bottomrule
\end{tabular}
\end{center}

%%%%%%%%%%%%%%%%%%%%%%%%%%%%%%%%%%%%%%%%%%%%%%%%%%%%%%%%%%%%%%%%%%%%%%%%%%%%%
%%%%%%%%%%%%%%%%%%%%%%%%%%%%%%%%%%%%%%%%%%%%%%%%%%%%%%%%%%%%%%%%%%%%%%%%%%%%%
%%%%%%%%%%%%%%%%%%%%%%%%%%%%%%%%%%%%%%%%%%%%%%%%%%%%%%%%%%%%%%%%%%%%%%%%%%%%%
\end{document}
