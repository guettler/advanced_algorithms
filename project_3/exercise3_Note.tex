\documentclass[11pt, notitlepage]{scrartcl}
\usepackage{graphicx}    % needed for including graphics e.g. EPS, PS
\usepackage{amsmath}
\usepackage[T1]{fontenc}
%\usepackage[utf8]{inputenc}
\usepackage[latin1]{inputenc}
\usepackage[german]{babel}
\usepackage{subfigure}
\usepackage{multirow} 
\usepackage{multicol}
\usepackage{booktabs} % for midrule
\usepackage{colortbl}
\usepackage{color}
\usepackage{hyperref}
\definecolor{Gray}{gray}{0.6}

\usepackage{tabularx}          
\newcolumntype{C}[1]{>{\centering\arraybackslash}p{#1}} 
\newcolumntype{R}[1]{>{\raggedleft\arraybackslash}p{#1}} 

\addtokomafont{captionlabel}{\bfseries} %references bf

\topmargin -1.5cm        % read Lamport p.163
\oddsidemargin -0.04cm   % read Lamport p.163
\evensidemargin -0.04cm  % same as oddsidemargin but for left-hand pages
\textwidth 16.59cm
\textheight 24cm 
\parskip 7.2pt           % sets spacing between paragraphs
\parindent 0pt     % sets leading space for paragraphs

%\usepackage{fancyhdr}
%\pagestyle{fancy}
%\fancyhf{}



%\renewcommand{\headrulewidth}{0.5pt}
%\fancyfoot[C]{\thepage}
%\renewcommand{\footrulewidth}{0.5pt}
% folder for images
%\graphicspath{{./img/}}
\begin{document}         


\title{Advanced Algorithms for Bioinformatics} 
\subtitle{Exercise 3: Compression with BWT}
%\author{ }
%\author{}
\author{Group 5: N. G"uttler, K. Liebers, F. Mattes} % lexicogrphic sorted
\maketitle

%%%%%%%%%%%%%%%%%%%%%%%%%%%%%%%%%%%%%%%%%%%%%%%%%%%%%%%%%%%%%%%%%%%%%%%%%%%%%
%%%%%%%%%%%%%%%%%%%%%%%%%%%%%%%%%%%%%%%%%%%%%%%%%%%%%%%%%%%%%%%%%%%%%%%%%%%%%
%%%%%%%%%%%%%%%%%%%%%%%%%%%%%%%%%%%%%%%%%%%%%%%%%%%%%%%%%%%%%%%%%%%%%%%%%%%%%
\section{Implementation}
%\textbf{q-gram index}\\


Tasks: (nur als Anhaltspunkte, d.h. am Ende weg damit!)

Mode c
\begin{itemize}
	\item   reads a single sequence from a fasta file
	   \item calculates the BWT of that sequence
    \item implements move-to-front encoding and Huffman coding to compress the BWT
    writes the Huffman code into an outfile 
	
\end{itemize}

Mode x
\begin{itemize}
	\item     reads a file containing the BWT compression of some sequence
	\item writes the uncompressed sequence into a fasta outfile. 
\end{itemize}

\textbf{Short introduction (TBD by?)}\\
\subsection{Mode c}




  
\textbf{BWT and move-to-front (short: 2-3 sentences TBD by Kurt)}\\
Skew7 from seqAn and algorithm as in script. Nothing special therefore only mention it and warn, that on linux we could not import seqan, therefore it could be problems by compiling.


\textbf{Huffman code: TBD by Kurt}\\
-Description of the used data structures and how to get the pathway(bitcode) of the characters i.e. leafs.\\
-outputfile?: 1-2 sentences

% #######################################################################################
\subsection{Mode x}
\textbf{Huffmancode to R ?  (short description 2-3 sentences TBD by Kurt, may be unnecessary)}\\
\textbf{Reverse MTF i.e. R to L (TBD by Nico)}\\
\textbf{LF mapping? i.e. BWT(L) to text (TBD by Nico)}\\

% #######################################################################################
\section{Results/Observations}
\subsection*{Comparison of the compression rate}


%%%%%%%%%%%%%%%%%%%%%%%%%%%%%%%%%%%%%%%%%%%%%%%%%%%%%%%%%%%%%%%%%%%%%%%%%%%%%
%%%%%%%%%%%%%%%%%%%%%%%%%%%%%%%%%%%%%%%%%%%%%%%%%%%%%%%%%%%%%%%%%%%%%%%%%%%%%
%%%%%%%%%%%%%%%%%%%%%%%%%%%%%%%%%%%%%%%%%%%%%%%%%%%%%%%%%%%%%%%%%%%%%%%%%%%%%
\end{document}
